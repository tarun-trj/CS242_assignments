\documentclass{article} 

\usepackage{amsmath}
\usepackage{hyperref}
\hypersetup{colorlinks=true, linkcolor=blue, urlcolor=blue, citecolor=blue} %specifying text colors for different link types

% Specifying dimensions and formatting for title and rule
\title{\noindent
\\Hello World!}
% Author and Date of document
\author{Tarun Raj}
\date{November 13, 2023}


\begin{document}
    \maketitle
    %Remove page numbers of current page
    \thispagestyle{empty}
    %Starting Text
    \section{Getting Started}
    \textbf{Hello World!} Today I am learning \LaTeX. \LaTeX{} is a great program for writing math. I can write in line math such as $a^2+b^2=c^2$. I can also give equations their own space: 
    \begin{equation} 
    \gamma^2+\theta^2=\omega^2
    \end{equation}
    % Maxwells Equations
    ``Maxwell's equations'' are named for James Clark Maxwell and are as follow:
    \begin{align}             
    \vec{\nabla} \cdot \vec{E} \quad &=\quad\frac{\rho}{\epsilon_0} &&\text{Gauss's Law} \label{eq:GL}\\      
    \vec{\nabla} \cdot \vec{B} \quad &=\quad 0 &&\text{Gauss's Law for Magnetism} \label{eq:GLM}\\
    \vec{\nabla} \times \vec{E} \quad &=\hspace{10pt}-\frac{\partial{\vec{B}}}{\partial{t}} &&\text{Faraday's Law of Induction} \label{eq:FL}\\ 
    \vec{\nabla} \times \vec{B} \quad &=\quad \mu_0\left( \epsilon_0\frac{\partial{\vec{E}}}{\partial{t}}+\vec{J}\right) &&\text{Ampere's Circuital Law} \label{eq:ACL}
    \end{align}
Equations \ref{eq:GL}, \ref{eq:GLM}, \ref{eq:FL}, and \ref{eq:ACL} are some of the most important in Physics.
\section{What about Matrix Equations?}
\begin{equation*}                                           % beginning of equation
\begin{pmatrix}                                             % beginning of matrix with rounded edges (pmatrix - parentheses)
a_{11}&a_{12}&\dots&a_{1n}\\                                % rows of matrix listed out
a_{21}&a_{22}&\dots&a_{2n}\\
\vdots&\vdots&\ddots&\vdots\\                               % dots to indicate and so on
a_{n1}&a_{n2}&\dots&a_{nn}
\end{pmatrix}                                               % pmatrix ends
\begin{bmatrix}                                             % beginning of matrix with straight edges (bmatrix - box matrix)
v_{1}\\                                                     % listing out rows of bmatrix
v_{2}\\
\vdots\\
v_{n}
\end{bmatrix}                                               % ending boxmatrix
=
\begin{matrix}                                              % beginning matrix without bounding edges
w_{1}\\                                                     % rows of the matrix
w_{2}\\
\vdots\\
w_{n}
\end{matrix}                                                % closing matrix
\vspace{+15mm}
\end{equation*}                                             % ending the equation started before
\end{document}
